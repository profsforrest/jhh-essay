%[MM SAYS:  IF WE WANT TO MAKE THIS SECTION SHORTER, WE COULD CUT EVERYTHING FROM HERE UP TO ``[UP TO HERE]'' (SEE BELOW)]

Inspired by Bellman \cite{Bellman1961} and
others, Holland explored the exploitation-versus-exploration tradeoff
via an idealized ``two-armed bandit'' problem.  Given a slot machine
with two arms, each of which has an unknown payoff probability, how
should you allocate $N$ trials (pulls) between the arms so as to
maximize your total payoff?  An example of an extreme \emph{exploitation} strategy would be to alternate between the arms until
one of them gives a payoff, and then allocate all future trials to
that arm alone.  Conversely, an example of an extreme \emph{exploration} strategy would be to randomly allocate the trials,
irrespective of the payoff rates you obtain.  Obviously, each of these
strategies is flawed.  What is an optimal strategy?
In \cite{Holland1973,Holland1975} Holland derived
an equation for such a strategy and argued that the optimal strategy
allocated trials to the observed best arm at a slightly higher rate than
exponential.  
% SF used 'argue' because some claim that his derivation was wrong.
He then extended this 
% Let $A$ denote the arm currently
%observed to have the higher payoff probability, and $B$ denote the
%other arm.  Holland showed that the optimal strategy (the one that
%yields maximal payoff, or minimal loss) is for the number of trials
%allocated to $A$ to grow slightly faster than an exponential function
%of the number of trials allocated to $B$.  Holland then showed that this
result to the \emph{multi-armed bandit} case.

Holland then made an analogy between multi-armed bandits and
populations undergoing adaptation: building blocks in a population
undergoing adaptation constitute the ``arms'' of a multi-armed bandit.
Each building block can be said to have a probability of ``payoff''
(i.e., contribution to a given individual's fitness).
%Evaluating an individual in an environment is
%like ``pulling the arms'' on a multi-armed bandit, where the ``arms''
%correspond to each of the building blocks making up that individual.
Thus, in the process of assigning a fitness to each individual in a
population, adaptive evolution can be seen as implicitly sampling the
many building blocks making up that collection of individuals.  The
average fitness over many individuals containing a particular building
block gives an estimate of that building block's payoff probability.

% The question of how to balance exploitation and exploration---how to
% optimally allocate trials to different arms based on their observed
% payoff---now becomes the question of how to optimally sample in the
% vast space of possible building blocks, based on their observed
% average fitness.  Of course adaptive evolution deals in populations of
% individuals, not building blocks.  There is no explicit mechanism for
% keeping statistics on building blocks' ``observed average fitness''.
% However, Holland's central idea here is that an approximation to
% optimal building-block sampling does indeed occur, as an emergent
% property of the population dynamics.

%Holland argued that an approximation to
%optimal building-block sampling actually occurs, as an emergent
%property of the population dynamics defined by a
% To show this mathematically, he defined an idealization of
%adaptive evolution: an algorithm he called a \emph{reproductive
% plan}.
To show this mathematically, Holland defined an idealization of adaptive evolution: 
a population-based stochastic process
operating on bit strings, involving the ``genetic'' operators of
 reproduction, fitness-based selection, crossover, mutation, and
 inversion.  
%Building blocks are formalized as \emph{
%   schemata}---patterns of bits within a string. (Here are two examples
% of schemas: ``all strings beginning with the bits 10'' or ``all
% strings that start with the pattern 1*1'', where * can be replaced by
% either 0 or 1).

% Holland's most important result was the following: He proved
% mathematically that his reproductive plan results---implicitly---in a
% near-optimal allocation of trials to schemata, thus optimizing the

% exploitation versus exploration balance.  Holland noted that the
% explicit act of assigning fitnesses to the individuals in a population
% was actually implicitly sampling a much larger collection of building
% blocks.  Holland termed this \emph{implicit parallelism}; the
% parallel leveraging of statistics from implicit sampling is a main
% strength of his population-based approaches to search.

[UP TO HERE]