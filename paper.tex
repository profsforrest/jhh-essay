\documentclass{sig-alternate}
\usepackage{color}

\newcommand{\red}[1]{\textcolor{red}{#1}}

\title{Adaptation Writ Large:\\ An Essay in Memory of John H. Holland}

\numberofauthors{2}
\author 
{\alignauthor
 Stephanie Forrest\\
 \affaddr{University of New Mexico}\\
 \affaddr{Santa Fe Institute}\\
 \email{forrest@cs.unm.edu}
 \alignauthor
Melanie Mitchell \\
 \affaddr{Portland State University}\\
 \affaddr{Santa Fe Institute}\\
 \email{mm@pdx.edu}
}

\begin{document}
\maketitle

% \begin{abstract}
% \input{abstract}
% \end{abstract}

\section{Introduction}

Professor John H. Holland passed away recently in Ann Arbor, MI, where
he had been on the University of Michigan faculty for over 50 years.
John, as he was known universally to his colleagues and students,
leaves behind a long legacy of intellectual achievements that is in
some ways difficult to pin down.  He is best known for the mechanisms
he proposed, especially genetic algorithms, but the motivations behind
the mechanisms are less well known but probably more important in the
long run.  These mechanisms did not always implement his theoretical
framworks perfectly, and in many cases too much effort was spent
trying to rescue the mechanisms and not enough in understanding the
underlying rationale.  In this short essay, we aim to remind people of
some of the key themes of his work, discuss the role they have played
in computer science, and highlight especially the ideas that we think
remain relevant to today's research agendas.  Many of the problems
that John identified and spent his life studying, such as [FILL IN
  HERE], are still poorly understood.

At a time, when computer science was crytstallizing into an
engineering discipline, John came out of the cybernetics era (dating
back to von Neumann, Wiener, Ashby, Turing) with a view of computation
as a broad interdisciplinary enterprise.  Thus, he became an early
proponent of interdisciplinary approaches to computer science and was
an active evangelist of what is now called computational thinking,
reaching out psychologists, economists, physicists, linguists,
philosophers and pretty much anyone he came in contact with.  As a
result, even though he received what was arguably the first Ph.D. in
our discipline, his contributions are much better known outside of
computer science than within.

Here we emphasize four recurrent themes of his work: adaptation as
exploitation and exploration, learning and credit assignment, models,
and the generic properties of complex adaptive systems.

\section{Adaptation as Exploitation and Exploration}

Although John Holland is best known for inventing genetic algorithms, his
lifelong motivation was much broader: to understand the
phenomenon of adaptation in all its guises.  Holland's interest in
developing a general theory of adaptation was spurred both by his
early work on computer models of Hebbian learning
\cite{RochesterHolland}.  This interest developed further with
Holland's extensive reading in evolutionary biology, economics, game
theory, and control theory.  Holland recognized that all these fields
in essence concern agents that must use information as it is gained from
uncertain, changing environments in order to continually improve their
performance.  Moreover, Holland recognized that in systems with
``adaptive'' agents of this kind, there is never a state of {\emph
  equilibrium} or a final {\emph optimum} configuration; rather, due
to the environment's ``perpetual novelty'' (as Holland termed it),
adaptation continues forever in an open-ended way.

Underlying Holland's theory of adaptation are three key ideas: (1) In
a population undergoing adaptation, individuals can be decomposed into
{\emph building blocks}---traits that are the ``atoms'' of an
individual's fitness or performance.  (2) Successful adaptation
requires the right tradeoff between {\emph exploitation}, in which
tried-and-true building blocks propagate in a population, and {\emph
  exploration}, in which new building blocks (or recombinations of
existing building blocks) arise and are subject to the test of the
environment. (3) Adaptive evolution (biological and otherwise) is a
population-based stochastic process that results in a nearly optimal
exploitation versus exploration tradeoff.   

Holland's early papers (e.g.,
\cite{OutlineLogicalTheory,OptimalAllocation}) and his influential
1975 book {\emph Adaptation in Natural and Artificial Systems}
\cite{ANAS} developed a general, formal setting in which these ideas
could be expressed mathematically.  Moreover, Holland stated and
proved a set of theorems that captured the third idea: that, given certain
assumptions, his model of adaptation, while explicitly modifying
populations of individuals, implicitly searched the space of possible
building blocks in a way that optimized the exploitation and
exploration balance.  (As an earlier part of this work, Holland
separately derived an equation for that optimal balance, via a
so-called ``multi-armed bandit'' model.)    

It was this formal setting and resulting theorems that led to the
invention of genetic algorithms, which featured stochastic population-based
search and recombination (or {\emph crossover}) as the most important
operation.  However, it is important to note that
the framework developed in \cite{ANAS} was more general than the
genetic algorithm and was created in order to develop an
interdisciplinary theory of adaptation, one that would inform biology,
say, as much as computer science \cite{ChristiansenFeldmanPaper}.  The
later, successful application of genetic algorithms to real-world
optimization and learning tasks was, for Holland, just icing on the
cake.

\section{Learning Via Credit Assignment and Rule Discovery}

\section{Modeling}

Modeling is central to Holland's theory of adaptive systems.  He posits that all adaptive systems create and use internal models to survive in their environments, through anticipation.  Models can be tacit and learned through evolutionary time, as in the case of bacteria swimming up a chemical gradient, or explicit and learned over a single lifespan as in the case of cognitive systems that incorporate experience into internal rule sets through learning.  In \underline{Induction}, Holland and his co-authors discuss how such internal models can be learned, without supervision, by combining environmental inputs with stored knowledge.  A key idea is that a model defines an equivalence relation over a set of environmental states, together with a set of transition rules, which are learned over time based on environmental inputs.  Models that form valid homomorphisms with the environment allow the system to make accurate predictions.  In his conception, the equivalence classes are initially very general, say 'moving object' and 'stationary object,' and over time through experience and learning they are specialized into more useful and precise subclasses, say 'insect' and 'nest', which form a default hierarchy and are associated with more specific transition rules.  Although the idea of default hierarchies is prevalent in many knowledge representation systems (LIST A FEW?), Holland made two key contributions, first by emphsizing homomorphisms as a formal way to evaluate model validity, and idea that dates back to Ross Ashby's \underline{An Introduction to Cybernetics} that Holland's student, Bernard Ziegler, developed into a formal theory of computer modeling and simulation.  Today, even though computational modeling is a major application area of computing,  essential to most natural sciences and engineering disciplines, these early homorphic theories of modeling stand as the most elegant approach we know of for characterizing when a model is consistent with the environment and how an intelligent agent, human or artificial, can update the model to better reflect reality.
Holland's second key contribution was describing a computational mechanism by which a cognitive system could iteratively build up a detailed and hierarchical model of its environment, as Melanie already discussed above.

Given that Holland believed that the ability to learn and manipulate internal models was essential for any adapting system, it is no suprise that he viewed modeling as essential for scientific inquiry.  
Today, we use computational models in several distinct ways---as a tool for analyzing data in statistical models, through simulation as a method for discovering new knowledge, and as a way to explain or understand how a system works.  This third use of models was dear to Holland's heart.  In his view, the key to science was understanding the mechanisms that cause a system to behave in a certain way, an aspiration that goes well beyond data fitting methods, which typically focus only on describing the aggregate behavior of a system.  For example, a statistical model that describes the boom and bust pattern of the stock market cannot address the underlying mechanisms that lead to these cycles, through the collective actions of many many individual buy/sell decisions.  Similarly, the genetic algorithms for which he is so famous provide a simple computational framework in which to explore the dynamics of Darwinian evolution and whether the basic mechanisms of variation, differential reproduction, and heredity are sufficient to account for the richness of our natural world.  This emphasis on exploratory models to build intuitions was an important theme of Holland's work, and he often quoted Eddington's remark on the occasion of the first experimental test of Einstein's theory of relativity: "The contemplation in natural science of a wider domain than the actual leads to a far better understanding of the actual."  Holland was interested in models that explored basic principles and mechanisms, even if they didn't make specific predcitions, models that could show generically how certain behaviors could be produced, for example how exponential growth might interact with finite resources.  Holland pioneered a style of modeling that has come to be known as 'individual based' or 'agent based,' in which every component of a system is represented explicitly and has state, e.g., every trader in a stock market system or every cell in an immune system model, and each agent had its own behaviors, which it could update (learn) over time.  These models are often defined over spatial structures, such as networks or simple grids, to capture the constraints of systems living under spatial constraints.  An agent-based model, then, encodes a theory about the mechanisms that are relevant for producting the behavior of interest.  Similar to expert systems, such models are especially useful for studying systems that don't have analytic mathematical descriptions and they can facilitate interdisicplinary collaborations because the underlying rules can be easily communicated.  It should be noted that Holland's view of modeling is by no means typical.  For example, in a textbook on computational modeling, the authors offer the following definition:  “Modeling is the application of methods to analyze complex, real-world problems in order to make predictions about what might happen with various actions.” [Shiflet and Shiflet, 2006].  This sort of perspective completely rules out the kind of modeling that Holland was most interested in, namely exploratory modeling.  Similarly, many dispute that models make any kind of scientific contribution:  “Models are metaphors that explain the world we don’t understand in terms of worlds we do.  They are merely analogies, provide partial insight, stand on someone else’s feet.  Theories stand on their own feet, and rely on no analogies.”  [Emanuel Derman, 2012].  [NOW WE NEED A STRONG FINISHING SENTENCE TO RESCUE JHH STYLE MODELING.]



\section{Complexity}

\section{Conclusion}

Introduced a few generations of students to computation in natural systems, an idea that today is better accepted.  His insights were deeper and more general than what often passes for work in biomimicry, e.g., for robots.

The ideas have had huge impact and should still be a beacon for research in intelligent and complex systems

John's personality and humanity is inextricably tangled up with his intellectual contributions.


\end{document}
