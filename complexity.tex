\section{Complex Adaptive Systems}

Holland was interested in a broad array of adaptive systems---immune systems, ecologies, financial markets, cities, and the brain---systems that are \emph{complex}.  In the 1980's, he teamed up with a small group of scientists, primarily physicists with a sprinkling of economists and biologists, to discuss what this wide swath of systems has in common.  The discussions helped define the intellectual mission of the Santa Fe Institute, the first institution dedicated to developing a science of complexity.  [add in how many complexity institutes there are worldwide and which others John played a hand in.]  Holland brought to these discussions his lifelong study of adaptation and a reminder that true theories about complexity would need to look deeper than phenomenological descriptions but also account for the  'how' and 'why' of these systems.
%describing phenomena like chaos, power laws, and ONE MORE, and acoount for the  
As the discussions matured, a consensus developed about the basic elements of complexity---systems that: (1)  are composed of many components with nontrivial (nonlinear) interactions; (2) behavior that emerges from the dynamics of the system, exhibiting higher-order patterns; (3) Scale, systems are nested and structure/behavior emerges at different scales, with some behavior being conserved across all scales and other behaviors changing at different scales, and (4) continual adaptation where systems adjust their behavioral rules through evolution and learning.  Although this is far from a formal definition of complex systems, most people working in the field today are interested in systems that have these properties.

To illustrate his ideas about the ubuiquity of adaptation, Holland developed two systems, ECHO and the SFI stock market.
Although less famous than genetic algorithms and learning classifier systems, ECHO was no less ambitious.  It's scope included the evolution of interacting populations of individual agents, which through competition and learning, discover symbiotic triads such as the famous ant-fly-caterpillar interaction,  the Wiksell triangle of trading relationships in economics, and the immune system's learned ability to distinguish 'self' from 'other.'  Without going into the details of the model [CITES HERE], ECHO formalized Holland's theories about complex adapting systems into a runnable computational system where agents evolved external markers (called tags) and internal preferences and used them to form higher level aggregate structures (trading relationships, symbiotic groups, trophic cascades, interdependent organizations etc.).  ECHO agents sometimes discovered mimicry to deceive competitors, and over time, the model developed increasingly complex structures and behaviors, often reproducing patterns observed in nature.  For example,  ECHO evolved a diversity of agent types, whose rank-frequency distribution closely parallels the well-known Preston curve in ecology, a quantiative statement of the adage that 'most species are rare'. Many of the insights behind this project are described in his book, [WHICH ONES]. The broad scope of this project was appealing to immunologists, economists, and evolutionary biologists alike, but it was the economists who took it one step further in the SFI stock market.

% Note: SF has the history backwards here.  We should correct that, but the story was so nice in this order

In the early 1990s, Holland teamed up with other SFI researchers, including two economists, to tackle the mismatch between predictions of the dominant theory in economics, the rational expectations hypothesis, and empirically observed stock market behaviors that deviated from the predicted equilibria solutions.  In brief, most economic theory of the day assumed that all participants in an economy or financial market are 100\% rational and act to maximize their individual gain.  We know, however, that real actors in real economies and markets are rarely rational, and financial markets often deviate from rationalilty, for example with speculative bubbles and crashes.  The SFI stock market project studied what happens in a model where rational traders are replaced by traders who learn to forecast stock prices over time.  The model was constructed to allow testing for the possible emergence of fundamental trading versus technical trading versus uninformed trading. The simulated market with (learning) trading agents was run many times and the dynamics of price and trading volumes were compared to observed patterns in real markets, replicating many of the features of real-life markets.  Although the model was primitive in many ways, it was wildly influential and led to many follow-on projects  It also demonstrated the important role that adaptation plays in complex systems and illustrated how Holland's theories of continual learning in response to intermittent feedbacks from the environment could be integrated into domain-specific settings.

%Blake LeBaron (economics);
% W. Brian Arthur (economics); 
% John Holland (psychology/EE/CS,
% and father of GAs);
% Richard Palmer (physics);
% Paul Taylor (computer science

%Many modeling issues not satisfactorily resolved by the SF-ASM model have been 
% taken up in later research (see Ref.[4]).

Holland's later books,  \emph{Emergence}, \emph{Signals and Boundaries}, and \emph{Complexity, A Summary} show how the theories of adaptation that he developed during the earlier part of his career fit into the larger landscape of complex systems research.   Holland's focus on how complex patterns emerge and change, rather than simply characterizing the patterns themselves (e.g., describing chaotic attractors or power laws) reflected his deterimination to 'get to the heart' of complex systems.  This determination 
represents the best of science.  Holland's willingness to tackle the hardest questions, develop his own formalisms, and use mathematics productively sets a high bar that we all should aspire to.


% Find the patterns.
% Why they occur is harder to say.  
% John focused on Mechanisms


%Ironically, Holland, didn't consider the computers of the day to be complex.

%The idea of a Nash equilibrium underlies much of economics; much of genetics, even today, considers only a single gene at a time; 

% The basic elements included systems with many autonomous components acting in parallel, a focus on the dynamics of these systems and their nonlinear intereactions, and emergent behavior.  The basic tools included nonlinear dynamical systems theory, information theory, and statistical mechanics [MELANIE, ADD MORE HERE].  Holland contributed a mature and generic view of adaptive systems, pointing out that the most interesting complex systems were continually adapting.  He was instrumental in defining the intellectual vision for the Santa Fe Institute, the first of what today number about [HOW MANY] such institutes worldwide devoted to the sciences of complexity.  





JOHN WAS INTERESTED IN ADAPTIVE SYSTEMS, LIKE CITIES, IMMUNE SYSTEMS, ECO-SYSTEMS, FINANCIAL MARKETS, E 

AND, I REMEMBER HAVING ARGUMENTS WITH HIM ABOUT WHETHER A COMPUTER COULD PROPERLY BE VIEWED AS A COMPLEX SYSTEM.  I SAID ‘YES’, AND HE SAID ‘NO’.   HIS POINT WAS THAT COMPUTERS WERE INCAPABLE OF OPEN-ENDED EVOLUTION AND EXPLORATION LIKE THESE OTHER SYSTEMS.   IN SOME WAYS, MY ENTIRE CAREER HAS BEEN DEVOTED TO SHOWING THAT COMPUTERS
HAVE ACQUIRED AT LEAST SOME OF THE PROPERTIES OF COMPLEX SYSTEMS THAT HE WAS SO INTERESTED IN.


